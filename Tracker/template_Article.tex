\documentclass[]{article}

%opening
\title{Quantum Kicked Rotor}
\author{Aditya Chincholi}

\begin{document}

\maketitle

\section{Introduction}

\section{Questions}
\begin{enumerate}
    \item What does kicking the rotor periodically have anything to do with a random walk?
    
    \item What is the analogy with Anderson localization? After all, Anderson localisation
    is about a diffusing wavefunction which encounters disorder in the form of passive scatterers
    and gets reflected/transmitted with certain probability. This transmission amplitude goes down
    exponentially with length of the sample. In contrast, in the kicked rotor we have an initial
    condition of uniform distribution in the position space. We have active "kicks" which pump
    energy into the system and these kicks strengths are pseudo-random apparently.
    
    \item If we take a quantum rotor and kick it with a truly random kick strength like $Ka_j$ at $t = j$
    where $a_j = 1$ with probability $p$ and $a_j = -1$ with probability $1-p$, would it also show initial
    diffusion and subsequent localisation?
\end{enumerate}

\end{document}
