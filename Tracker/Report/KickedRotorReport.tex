\documentclass[twocolumn]{report}
\usepackage[margin=0.75in]{geometry}
\usepackage[utf8]{inputenc}
\usepackage{amsmath}
\usepackage{amssymb}
\usepackage{braket}
\usepackage{graphicx}
\usepackage{subcaption}
\usepackage{float}
% \usepackage[parfill]{parskip}
\usepackage{biblatex}
\usepackage{hyperref}
\addbibresource{citations.bib}

\hypersetup{
    pdftitle={Quantum Kicked Rotor Systems},
    pdfauthor={Aditya Chincholi},
    pdfsubject={Semester Project Report},
    bookmarksnumbered=true,
    bookmarksopen=true,
    bookmarksopenlevel=1,
    colorlinks=false,
    pdfstartview=Fit,
    pdfpagemode=UseOutlines,
    pdfpagelayout=TwoPageRight
}

% Title Page
\title{Quantum Kicked Rotor Systems}
\author{Aditya Chincholi}

\begin{document}
\maketitle

\onecolumn
\begin{abstract}
    We discuss kicked rotor systems and the properties of dynamical
    localisation in these systems, particularly their robustness to
    noise. We also discuss the quasiperiodic kicked rotor and the
    change in its properties as it crosses the critical point of the
    metal-insulator transition, particularly the change in the spectrum.
    We present results on the evaluation of the bipartite entanglement
    entropy in the 3d quasiperiodic kicked rotor system.
\end{abstract}

\tableofcontents


\twocolumn
\chapter{Introduction}
% We give a brief introduction to the kicked rotor system
% and all it's variants.
The quantum kicked rotor is defined by the Hamiltonian:

\begin{equation}
H = \frac{p^2}{2} + k cos(\theta) \sum_{n \in \mathbb{Z}} \delta(t - n\tau)
\end{equation}

This classical analogue of this system reduces to the Chirikov standard map
\cite{stockmann}:

\begin{align}
    p_{n+1} = p_n + k sin(\theta_n) \\
    \theta_{n+1} = \theta_n + p_{n+1} \label{eq:classicalmap}
\end{align}

This map is chaotic and shows a great deal of complexity, but we will only
deal with the quantum version here. The key point for us, is that this
classical map displays diffusion in the angular momentum space. This can be
seen qualitatively by looking at \ref{eq:classicalmap}. When $p_n$ becomes
larger than $2\pi$, which will happen after a few kicks if $k$ is large,
then the successive $\theta_n$ will become uncorrelated. As the sign of
$sin(\theta_n)$ becomes random, the sequence ${p_n}$ describes a random walk.
This can be backed by a quantitative calculation which shows that
the distribution $f_N(p)$, i.e. the distribution of momentum after N steps
wrt different initial conditions ($\theta_0$), is well approximated by a
gaussian distribution with a diffusion constant \cite{stockmann}:
\begin{equation}
    D = \frac{k^2}{2}
\end{equation}

As a time dependent problem, the quantum kicked rotor has no energy eigenstates.
However, since it is a periodic kicking potential, we can use floquet operators
to simulate it.

\begin{align}
    F &= \lim_{\epsilon \to 0} \int_{0}^{\tau + \epsilon} exp(-iHt/\hbar) dt\\
    &= exp\left(\frac{-i}{\hbar} cos\ \theta\right)
    exp\left(\frac{-i\tau}{2\hbar}p^2\right)
\end{align}

The floquet operator allows us to obtain the state of the system at $t = n\tau$.
Working in the eigenbasis of the (angular) momentum operator
$p \ket{n} = \hbar n \ket{n}; \braket{\theta | n} = e^{in\theta}$, we get:

\begin{equation}
    \braket{m | F | n} = exp\left(-\frac{i \tau}{2\hbar}m^2\right) i^{n-m}
    J_{n-m}\left(\frac{k}{\hbar}\right)
\end{equation}

We can use this expression for the floquet matrix to simulate the quantum
kicked rotor system as $F^N \ket{\psi(0)} = \ket{\psi(N\tau)}$. Even though
the system doesn't have stationary states, we can obtain floquet eigenstates
by diagonalising the floquet operator.

An important variant of the kicked rotor system is the quasiperiodic kicked
rotor, which is discussed further in the report. It is given by the
Hamiltonian:

\begin{equation}
H = \frac{p^2}{2} + \mathcal{K}(t) cos(\theta)\sum_{n \in \mathbb{Z}}
\delta(t - n\tau)
\end{equation}

where $\mathcal{K}(t) = k(1 + \epsilon cos(\omega_2 t + \phi_2)
cos(\omega_3 t + \phi_3))$.

\chapter{Readings and Work Done}
\section{Localisation}
The quantum kicked rotor also shows the same diffusion mechanism as the
classical analogue in the beginning. If we take the initial state as $\ket{0}$
which has a uniform distribution of $\theta's$, then the system shows a
gaussian shape in the momentum space. But after a certain time, the
diffusion is suppressed by quantum effects which lead to an exponential
localisation in the momentum space distribution around $\ket{0}$. This
phenomenon is called `dynamical localisation'. It is analogueous to the
Anderson localisation found in tight-binding systems.

We now simulate the system using $\ket{\psi(N\tau)} = F^N \ket{0}$. The
results are shown in Figure \ref{fig:kickedrotorplot}. The figure shows the
state of the system as after different number of timesteps. The initial
stages ($t = 100$) show the classically expected gaussian profile in the
momentum space probability distribution. But after that the quantum effects
begin to dominate and we obtain a exponentially linear profile i.e. $P(p)
\sim e^{-p/p_{loc}}$.

Though it is hard to see on the log scale, $\langle p^2 \rangle$ shows a
lineary increasing profile during the diffusion phase (this is the same as
the classical case) but then saturates as localisation occurs at longer time
scales.

So why does localisation occur? Qualitatively speaking, after certain
number of timesteps we see destructive interference happening for
the transition amplitudes (i.e. the probability of $\ket{m} = F\ket{n}$)
when we calculate the expectation $\langle p^2 \rangle$. For a more
quantitative argument, the reader is encouraged to see section 4.2.1 of
\cite{stockmann}.

A different way to see the emergence of localisation is through a nonlinear
transformation that connects the kicked rotor system with the Anderson model.
The next section will cover the Anderson model in more detail. Here we just
briefly outline the mapping, for more details see \cite{stockmann}.
Consider a 1d quantum kicked rotor of the form:
\begin{equation}
    H = K(p) + V(\theta)\sum_{n} \delta(t - n)
\end{equation}
Then we obtain the floquet operator:
\begin{align}
    F &= e^{-iV(\theta)} e^{-iK(p)} \\
    F_{nm} &= e^{-iK(m)} J_{n-m} \\
    \text{where } & J_n = \frac{1}{2\pi} \int_0^{2\pi} e^{-iV(\theta)}
    e^{in\theta} d\theta
\end{align}
We consider the eigenvector $\mathbf{a}$ of $F$ with eigenphase $\phi$
and use the mapping $W(\theta) = -tan(V(\theta) / 2)$ and $\bar{\mathbf{a}}
= (e^{i(\phi - K)} + 1)\mathbf{a}$. Then we obtain:
\begin{equation}
    tan\left(\frac{\phi - K}{2}\right) \bar{\mathbf{a}} + W\bar{\mathbf{a}} = 0
\end{equation}
Taking a fourier transform of all the components wrt $\theta$
\begin{equation}
    \sum_{k \neq n} W_{n-k} \bar{a_k} + E^0_n \bar{a_n} = E \bar{a_n}
\end{equation}
where $E_n^0 = tan((\phi - K(n)) / 2)$ and $E = -W_0$. This is the Anderson
tight-binding model with bond strengths given by the $W_{n-m}$'s and site
energies given by $E^0_n$'s.

% \onecolumn
\newpage
\onecolumn
\begin{figure}[H]
    \centering
    \begin{subfigure}[b]{0.5 \textwidth}
        \centering
        \includegraphics[width=0.8 \linewidth]{img/quantumkickedrotor_T100.png}
        \caption{After 100 timesteps.}
    \end{subfigure}%
    \begin{subfigure}[b]{0.5 \textwidth}
        \centering
        \includegraphics[width=0.8 \linewidth]{img/quantumkickedrotor_T200.png}
        \caption{After 200 timesteps}
    \end{subfigure}%
    \newline
    \begin{subfigure}[b]{0.5 \textwidth}
        \centering
        \includegraphics[width=0.8 \linewidth]{img/quantumkickedrotor_T500.png}
        \caption{After 500 timesteps.}
    \end{subfigure}%
    \begin{subfigure}[b]{0.5 \textwidth}
        \centering
        \includegraphics[width=0.8 \linewidth]{img/quantumkickedrotor_T750.png}
        \caption{After 750 timesteps.}
    \end{subfigure}
    \caption{Evolution of the kicked rotor system with $k = 5$, $\hbar = 1$
and $\tau = 1$ in a basis of $\{\ket{-1000},...,\ket{1000}\}$. The top
left plot is $[p^2]$ vs $t$; top right is $log_{10}(P(p = n\hbar))$ vs
$n$ i.e the log of the probability distribution in momentum space and the
bottom left plot is the variance of momentum square $[(\Delta(p^2))^2]$ vs
time $t$.}
    \label{fig:kickedrotorplot}
\end{figure}

\begin{figure}
    \begin{subfigure}[b]{0.5 \textwidth}
        \includegraphics[width=0.8 \linewidth]{img/quantumkickedrotor_basecase.png}
        \centering
        \caption{Base case with no noise.}
    \end{subfigure}
    \begin{subfigure}[b]{0.5 \textwidth}
        \centering
        \includegraphics[width=0.8 \linewidth]{img/quantumkickedrotor_deltak0.01.png}
        \caption{$\delta k = 0.01$(0.2 \%)}
    \end{subfigure}
    \begin{subfigure}[b]{0.5 \textwidth}
        \centering
        \includegraphics[width=0.8 \linewidth]{img/quantumkickedrotor_deltak0.1.png}
        \caption{$\delta k = 0.1$ (2 \%)}
    \end{subfigure}
    \begin{subfigure}[b]{0.5 \textwidth}
        \centering
        \includegraphics[width=0.8 \linewidth]{img/quantumkickedrotor_deltak0.5.png}
        \caption{$\delta k = 0.5$ (10 \%)}
    \end{subfigure}
    \begin{subfigure}[b]{0.5 \textwidth}
        \centering
        \includegraphics[width=0.8 \linewidth]{img/quantumkickedrotor_deltatau0.01.png}
        \caption{$\delta \tau = 0.01$ (1 \%)}
    \end{subfigure}
    \begin{subfigure}[b]{0.5 \textwidth}
        \centering
        \includegraphics[width=0.8 \linewidth]{img/quantumkickedrotor_deltatau0.1.png}
        \caption{$\delta \tau = 0.1$ (10 \%)}
    \end{subfigure}
    \caption{Evolution of the quantum kicked rotor system with a perturbative
    noise added to either $k$ or $\tau$ at each stage. All runs have the base
    values $k = 5$, $\tau = 1$, $\hbar = 1$ and are run for either 1000
    timesteps or 500 timesteps on a basis of states
    $\{\ket{-1000}, ...\ket{1000}\}$. Top left is average sq.
    momentum $[p^2]$, bottom left is variance of sq. momentum
    $[(\Delta(p^2))^2]$ and top right is the momentum distribution
    $log_{10}(P(p = n\hbar))$ vs $n$}
    \label{fig:robustnessplot}
\end{figure}

\twocolumn
\section{Anderson Localisation}
We now take a short detour to explain the phenomenon of Anderson Localisation.
The contents of this section are largely taken from \cite{muller_disorder_2016}.

Consider a non-relativistic particle or gaussian wave packet propogating
through a channel. If the channel has a constant potential throughtout
then the wave will propogate through it unimpeded performing ballistic
motion. If the channel has a constant potential with a small noise
term, then the potential landscape will look like a series of speckles
on an otherwise flat surface. Let us assume these speckles are well
separated and look more like spikes rather than shallow hills.

We have three length scales here, $l$ the
spacing between these speckles, $\delta$ the width of the speckle and
$\lambda$ the de Broglie wavelength of the particle. We assume $\delta <<
\lambda << l$ and thus, the particle sees the speckles as well-separated
$\delta$-spikes.

Each of these speckles - wells and peaks - acts as a
scatter for our particle. Quantum mechanically, each of them has a
finite, non-zero probability of both reflecting and transmitting the
particle.

Suppose then that our particle hits scatterer 1 and gets
transmitted with some probability. It then performs ballistic motion and
hits scatterer 2. Again there is some probability of transmission and
reflection. The particle may get transmitted immediately, or it may
undergo reflection twice (a complete internal reflection) and then get
transmitted. It may undergo a complete internal reflection multiple times
before crossing the scatterer. This leads to the net transmission
probability being dependent on the phase difference accumulated over
a complete internal reflection.

The net transmission probability can
be found by multiplying the transfer matrices of the two scatterers and
obtaining an overall transfer matrix. This yields the expression:

\begin{equation}
    T_{12} = \frac{T_1 T_2}{1 + \sqrt{R_1 R_2} e^{i\theta}}
\end{equation}
where $T_i$ and $R_i$ denote the transmission and reflection
probabilities of the $i$th scatterer, $\theta$ is the phase accumulated
in a complete internal reflection.

This phase is distributed randomly irrespective of the distribution of
distances between the scatterers as $\lambda << l$ and thus, we may
assume a uniform distribution for $\theta$.

Decoherence can occur if during the ballistic part, the particle couples
to an external degree of freedom. In such a scenario, we can assume a
uniform distribution for $\theta$ and obtain an average transmission
probability. In doing this, we are effectively killing any interference
effects as we are considering the phase to be completely scrambled by the time
it reaches scatterer 2. The calculations then yield $\langle T_{12} \rangle
= T_1 T_2 / (1 - R_1 R_2)$. For a channel of length $L$ and nearly uniform
scatterer density $n$, this gives us $R/T \propto nL$ which is Ohm's law.

But things
are much more interesting if we have phase-coherent transmission through
the channel. We must then find a quantity that is additive for such a
transmission and average over it. $\kappa = - ln\ T$ is such a quantity.
It exhibits the property $\langle ln\ T_{12} \rangle = ln\ T_1 + ln\ T_2$ and
due to this, it displays self-averaging.

Therefore, for a channel of
length $L$, $|\langle ln\ T \rangle| \propto nL$. Since $T < 1$, $ln\ T < 0$
and thus we have $exp(\langle ln\ T \rangle) \sim exp(-L/\xi_{loc})$ where
$\xi_{loc}$ is called the localisation length. This exponential localisation
in the absence of absorption is a hallmark of strong localisation.

Here,
$ln\ T$ exhibits a normal distribution and its peak corresponds to the most
likely value $T_{typ} = exp(\langle ln\ T \rangle) = e^{-L/\xi_{loc}}$. This
phenomenon is called Anderson localisation.

In 3D, the situation becomes
more complicated and the system can either allow transmission or show
localisation based on disorder strength. These two regimes are separated by
a 2nd order phase transition known as the Anderson transition or the
metal-insulator transition, metal referring to transmission and insulator
referring to localisation of the particle.

Even though Anderson originally
introduced it in the form of a tight-binding model with discrete spatial
sites, our presented model also exhibits the same behaviour and thus has
been used in this report.

\section{Robustness}
We now consider the question of how robust the phenomenon of dynamical
localisation is to noise in the kicking potential. For the Anderson model,
the localisation is destroyed mainly if there are long-range correlations
in the scattering peaks. But we know that the mapping between the Anderson
model and the kicked rotor is a nonlinear mapping and hence, it may show
a very different behaviour to noise in the kicks.

We investigate the robustness of the kicked rotor model by introducing noise
in two different ways:
\begin{enumerate}
    \item Through kick strengths $k$: At each time step we add a small noise
    term to the kick strength $k$. So $F(k+\delta k)\ket{\psi((N-1)\tau)} =
    \ket{\psi(N\tau)}$.

    \item Through kicking periods $\tau$: At each timestep the kick is
    delivered at time $t = N\tau + \delta \tau$.
\end{enumerate}

This was implemented numerically in the same way as quantum kicked rotor,
the only difference being that a new floquet matrix must be generated at
each timestep. In the implementation, there is only one point of note - the
perturbation in the kicking period must be compensated for on the next step
in order to keep the kicking times of the form $t = N\tau + \delta \tau$.

The results are shown in Figure \ref{fig:robustnessplot}. The base case
shows saturation in the average sq. momentum and highly linear plots
for the log of the momentum distribution - both characteristic of
localisation. However, the localisation gets destroyed on adding noise
to the kicking strength. The effect is relatively weak in the sense that
the noise is nearly 10\% of the base kicking strength when we start seeing
the localisation completely disappear. At 2\% of the base kicking strength,
we only see the diffusive behaviour in the variance of sq momentum and
weakly in the average sq. momentum, the momentum distribution remains
largely localised even after a 1000 timesteps. At 0.1\% of the base
kicking strength, no effects are visible. This indicates that the
system is slightly robust to perturbation in the kicking strength.
However, the system shows high sensitivity to perturbations in the time
period of the kicking field. Even at noise of scale of 1\% of $\tau$,
the localisation completely breaks down and diffusive behavour is
observed. This shows that the localisation relies on a delicate
destructive interference.

\section{Anderson Transition}
The Anderson transition is a second order phase transition occuring in the
tight-binding Anderson model of dimension $d > 2$ \cite{muller_disorder_2016}.
The system shows metallic conduction i.e. delocalised states for small values
of disorder strength and insulator-like behaviour - localisation for larger
values of disorder strength. This phenomenon can be investigated analytically
through the use of a scaling function $\beta$ which describes the behaviour
of the (dimensionless) conduction as the length $L$ of the sample changes.
This is described in many places such as \cite{muller_disorder_2016}. We
shall not be pursuing that line of attack here.

We have already show the relation between the kicked rotor systems and the
Anderson tight-binding models in previous sections. Now we consider a
particular case of this. We consider the quasiperiodic kicked rotor:
%
\begin{equation}
    H = \frac{p^2}{2} + \mathcal{K}(t) cos(\theta) \sum_n \delta(t - n\tau)
\end{equation}
%
where $\mathcal{K}(t) = 1 + \epsilon cos(\omega_2 t + \phi_2)
cos(\omega_3 t + \phi_3)$. The dynamics of this 1d system is identical
to the following 3d kicked rotor system:
%
\begin{align}
    H_{3d} &= \frac{p_1^2}{2} + p_2 \omega_2 + p_3 \omega_3\\
    &+ k cos(\theta_1) (1 + \epsilon cos(\theta_2) cos(\theta_3))
    \sum_n \delta(t - n\tau)\\
    \text{with } &\ket{\psi_{3d}(0)} = \ket{\psi_{1d}(0)}
    \ket{\theta_2 = \phi_2}\ket{\theta_3 = \phi_3}
\end{align}
%
If we consider the fourier transform of the eigenvalue equation
for the floquet operator of this Hamiltonian i.e.
$F_{3d}\ket{\phi_\omega} = e^{i\omega}\ket{\phi_\omega}$.
Taking a fourier transform of this equation we obtain:
%
\begin{equation}
    \epsilon_{\mathbf{m}} \Phi_{\mathbf{m}} + \sum_{\mathbf{r} \neq 0}
    W_{\mathbf{r}}\Phi_{\mathbf{m - r}} = -W_0 \Phi_{\mathbf{m}}
\end{equation}
%
where $\mathbf{m} = (m_1, m_2, m_3)$ and $\mathbf{r}$ label sites on
a 3d lattice and the $\Phi_{\mathbf{m}}$ are related to the fourier
components of the floquet eigenstate $\ket{\phi_\omega}$
\cite{fishman_chaos_1982}\cite{lemarie_universality_2009}.

The quasiperiodic kicked rotor is therefore expected to show a
phase transition from delocalised to localised states as the
the parameters are varied and certain conditions are satisfied.
In particular, the quadruplet $(\hbar, \omega_2, \omega_3, 2\pi)$
should be an incommensurate quadruplet ($\hbar$ is the effective
Planck's constant given by $[\theta, p] = i\hbar$). The transition
is observed as we increase $K$ and $\epsilon$. In particular, we use
a set of values $\hbar =  2.85$, $\omega_2 = 2\pi\sqrt{5}$,
$\omega_3 = 2\pi\sqrt{13}$, $K = 6.24 \to 6.58$ and
$\epsilon = 0.413 \to 0.462$. \cite{lemarie_universality_2009}

This phase transition has been studied in detail and the author
sought to reproduce the results from \cite{lemarie_universality_2009}.
\cite{lemarie_universality_2009} uses single parameter scaling theory
to utilise simulation data to evaluate properties of the critical
point of the transition in order to mitigate finite time effects.
The quantity $\Lambda = \langle (p/\hbar)^2 \rangle t^{-2/3}$ is
the relevant scaling parameter for this transition. The average in
$\langle (p/\hbar)^2 \rangle$ is an average over initial conditions
i.e. over $(\phi_1, \phi_2)$.

We could not however, reproduce the scale of the computation
done in the paper as it requires computational resources not
available to us (40,000 timesteps are a tad bit out of our reach
we must say). Nevertheless, we have presented the results
from this computation done to best of our resources in Figure
\ref{fig:scalingparam}.

\begin{figure}[t]
    \includegraphics[width=\linewidth]{img/quasiperiodic_scalingparam.png}
    \caption{The left plot shows the variation of the scaling parameter
    $\Lambda$ with k. Each of the lines represents one timestep, darker
    purple lines are from later timesteps while orange ones are earlier.
    The point of intersection is the critical point. The transition is known
    to occur at $k_c = 6.36$ \cite{lemarie_universality_2009}. The right plot
    shows how $\Lambda$ changes with time for different values of k sampled.}
    \label{fig:scalingparam}
\end{figure}

\section{Bipartite Entanglement}
The 1d quasiperiodic kicked rotor shows the dynamics of the 3d kicked
rotor for a particular value of $(\phi_2,\phi_3)$. It doesn't emulate
the full rotor as it lacks the degrees of freedom. Particularly missing
from the 1d case is the entanglement between the 1st momentum space and
the 2nd, 3rd (quasi)momentum spaces. That the floquet operator entangles
the spaces is evident from the calculation of the floquet operator:

\begin{align}
    &F_{3d} \\
    &= e^{-i K cos\theta_1 (1 + \alpha cos\theta_2 cos\theta_3) / \hbar}
    e^{-i (p_1^2 / 2 + p_2 \omega_2 + p_3 \omega_3)/\hbar} \\
    &\bra{\mathbf{m}} F_{3d} \ket{\mathbf{n}} \\
    &= \bra{m_1, m_2, m_3}F_{3d}\ket{n_1, n_2, n_3} \\
    &= \bra{\mathbf{m}}
    e^{-i K cos\theta_1 (1 + \alpha cos\theta_2 cos\theta_3) / \hbar}
    \ket{\mathbf{n}} \\
    &\ \ \ \ e^{-i (\hbar n_1^2 / 2 + n_2 \omega_2 + n_3 \omega_3)} \\
    &= \bra{\mathbf{m}} \int_{[0,2\pi]^3} d^3\theta \
    e^{-i K cos\theta_1 (1 + \alpha cos\theta_2 cos\theta_3) / \hbar}
    \ket{\boldsymbol{\theta}}\\
    &\ \ \ \ \braket{\boldsymbol{\theta}|\mathbf{n}}
    e^{-i (\hbar n_1^2 / 2 + n_2 \omega_2 + n_3 \omega_3)} \\
    &= \int_{[0,2\pi]^3} d^3\theta \ \braket{\mathbf{m}|\boldsymbol{\theta}}
    e^{-i K cos\theta_1 (1 + \alpha cos\theta_2 cos\theta_3) / \hbar}\\
    &\ \ \ \ \braket{\boldsymbol{\theta}|\mathbf{n}}
    e^{-i (\hbar n_1^2 / 2 + n_2 \omega_2 + n_3 \omega_3)} \\
    &= e^{-i (\hbar n_1^2 / 2 + n_2 \omega_2 + n_3 \omega_3)}
    \frac{1}{(2\pi)^3} \\
    &\ \ \int_{[0,2\pi]^3} d^3\theta \
    e^{-i(\mathbf{m} - \mathbf{n})\cdot \boldsymbol{\theta}}
    e^{-i K cos\theta_1 (1 + \alpha cos\theta_2 cos\theta_3) / \hbar}\\
    &\approx e^{-i (\hbar n_1^2 / 2 + n_2 \omega_2 + n_3 \omega_3)}
    DFT_N(f(\boldsymbol{\theta}))[\mathbf{m} - \mathbf{n}]\\
    &\text{where } \nonumber\\
    &f(\boldsymbol{\theta}) =
    e^{-i K cos\theta_1 (1 + \alpha cos\theta_2 cos\theta_3) / \hbar}
\end{align}
%
The fourier transform integral clearly mixes up the $\theta_1$, $\theta_2$
and $\theta_3$ terms and hence leads to entanglement. Therefore, we
should be able to measure the entanglement between the subspaces
$\mathcal{H}_1$ and $\mathcal{H}_{2,3}$. And this is what we attempted
to accomplish.

However, several issues arose in doing this. The first
and foremost issue was that we need to simulate a 3d quantum system in
order to calculate the bipartite entanglement entropy between
$\mathcal{H}_1$ and $\mathcal{H}_{2,3}$. Therefore, the memory required
and the computational complexity for computing the floquet and
density matrices scales as $N^6$ where $N$ is the size of the momentum
eigenbasis being used. The scale of the matrix also requires that the
the calculations be done with ``double'' precision, otherwise it yields
underflow errors as some of the fourier transform values are extremely
low off the diagonal. Since our basis states are very limited, the
approximation of the floquet operator is not very good. A significant
portion of calculated eigenvalues of the matrix have absolute values
less than even 0.9. This varies for different parameter values but is
highest for the higher k values. The calculations are also quite slow
to process, however, this was tackled by using C-extensions and parallelisation
via Intel OpenMP. As a result, our calculations are not
reliable, and therefore, we do not present our numerical analysis here.

\section{Spectral Analysis}
The theory of random matrices has been studied extensively by other people.
In this section, we draw from this and try to understand the spectral
structure of the floquet matrix. As the floquet operator elements consist
of the fourier transform of a kicking function with a pseudo-random strength
(at least for longer time scales). So we expect that the elements are randomly
distributed. Consequently, we would like to analyse it as a random matrix.

Now random matrix theory makes predictions about the eigenvalue spectrum
of the any matrix whose elements are sampled from a gaussian distribution.
Depending on the symmetries of the matrix, we can divide it into 3 different
ensembles: GUE (Gaussian Unitary Ensemble), GOE (Gaussian Orthogonal Ensemble)
and GSE (Gaussian Symplectic Ensemble).

To get an easily observable quantity, we consider the level spacings in the
spectrum of these matrices. These theories were made from the viewpoint of
analysing the Hamiltonian matrix of different quantum systems. Therefore,
the analysed quantity consists of the difference of consecutive eigenvalues
i.e. $d_i = \lambda_{i+1} - \lambda_{i}$. Since the Hamiltonian is hermitian, the
$\lambda_i$'s are real and thus we can talk about consecutive eigenvalues.
We define $P(s)$ as the probability distribution of the $s_i = d_i / \bar{d}$
where $\bar{d}$ is the mean level spacing. For systems where the local
density of states varies a lot, an `unfolding' procedure may be required to
obtain this. We will not need it and hence, do not describe it.

It can be shown that if the classical analogue of the system is integrable,
then the eigenvalues of the matrix are uncorrelated and thus the spacings
form a Poisson distribution \cite{stockmann}\cite{mehta_random_2004}. The
systems which have a chaotic classical analogue will fall into one of the 3
mentioned ensembles: GOE (if the system has time inversion invariance and
and rotational invariance), GSE (if the system is time inversion invariant
but not rotationally invariant) and GUE (in all other cases). A complicated
but exact distribution function can be obtained for the level spacings of
all of these \cite{mehta_random_2004}, but Wigner derived a simple formula
from the $2 \times 2$ matrix case which holds reasonably well, especially
for large N. These are listed below \cite{stockmann}:

\begin{align}
    P_{int}(s) &= exp(-s)\\
    P_{GOE}(s) &= \frac{\pi}{2} s\ exp\left(-\frac{\pi}{4}s^2\right)\\
    P_{GUE}(s) &= \frac{32}{\pi^2} s^2\ exp\left(-\frac{4}{\pi}s^2\right)\\
    P_{GSE}(s) &= \frac{2^{18}}{3^6 \pi^3} s^4\ exp\left(-\frac{64}{9\pi}s^2\right)
\end{align}

However, our kicked rotor systems have time dependent Hamiltonians, and
therefore we cannot apply this analysis to it. However, it turns out that
we can use the same strategy on the eigenenergies. These are the phase
angles of the eigenvalues of the floquet matrix. It turns out that their
distribution is nearly identical to the guassian ensembles' eigenvalues.
Therefore, there exist analogues of each of the ensembles called COE
(Circular Orthogonal Ensemble), CUE (Circular Unitary Ensemble) and
CSE (Circular Symplectic Ensemble). Since the space of these phases is
compact $[0, 2\pi)$, we also avoid complications with unfolding as the
density of states is uniform $\frac{N}{2\pi}$ ($N$ being the number
of eigenphases). Correspondingly, we define
\begin{equation}
    s = \frac{2\pi}{N} (\phi_{i+1} - \phi_i)
\end{equation}
where $\phi_i$ are the eigenphases of the floquet operator arranged in
increasing order. We can also define the consecutive level spacing ratio
distribution $P(r)$ where $r$ is defined as:
\begin{equation}
    r_i = \frac{s_{i+1}}{s_i}
\end{equation}
Now using the same techniques as used in deriving the Wigner surmises,
good approximations can be found for the ratio distributions from the
$3 \times 3$ matrix case. The ratio distribution also doesn't require
unfolding of any kind. The surmises takes the form\cite{atas_distribution_2013}:
\begin{equation}
    P_W(r) = \frac{1}{Z_\beta}
    \frac{(r + r^2)^\beta}{(1 + r + r^2)^{1 + 3\beta/2}}
\end{equation}
where $\beta$ is the Dyson index which is 1 for COE, 2 for CUE and 4
for CSE. The $Z_\beta$ are normalisation constants which can be found
in \cite{atas_distribution_2013}.

For the purposes of this project, attempts were made to calculate the
level spacing ratio distributions for the quasiperiodic kicked rotor
systems. However, the same problems as the last section still plague
us as the approximating floquet matrix has a maximum of 21 basis states
in each of the three directions owing to computational resource limitations.
Therefore, those results are not presented here.

\chapter{Epilogue}
\section{Future Prospects}
\begin{enumerate}
    \item The calculations for metal-insulator transition properties
    in the quasiperiodic kicked rotor can be enhanced through C-extensions
    and parallelisation as done for the 3d quasiperiodic kicked rotor.

    \item The calculations for the bipartite entanglement entropy can
    be repeated on a workstation with better computational resources
    and the same holds for the spectral analysis.

    \item One can try and investigate kicked rotor systems with
    perturbative driving such as $H = p^2/2 + k cos(\theta)\sum_{n}
    \delta (t - n) + \lambda A cos(l\theta)$ where $A << V$. Such a
    small driving term may be useful in using the system as a
    quantum computer if it conserves the localisation. If we take a
    small value of $l$, we may be able to use the adiabatic theorem
    to change the phase of system to attain some optimal configuration
    over several iterations (such as in Grover's algorithm) while
    maintaining localisation.
\end{enumerate}

\section{Conclusion}
We have discussed in some detail an overview of the phenomena that occur
in the quantum kicked rotor system and its variants. We have presented
numerical analysis of the kicked rotor and its robustness to noise in the
kicking strength and the time period of kicking. We have also presented
relevant background for the calculation of the bipartite entanglement
entropy between the first and the remaining momentum spaces and for the
spectral analysis of the floquet matrices of the kicked rotor systems.

\printbibliography

\end{document}
