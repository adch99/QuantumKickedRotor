\documentclass[twocolumn]{report}
\usepackage{amsmath}
\usepackage{amssymb}
\usepackage{braket}
\usepackage{graphicx}
\usepackage{biblatex}
\addbibresource{citations.bib}

% Title Page
\title{Quantum Kicked Rotor Systems}
\author{Aditya Chincholi}


\begin{document}
\maketitle

\onecolumn
\begin{abstract}
\end{abstract}
\twocolumn
\chapter{Introduction}
% We give a brief introduction to the kicked rotor system
% and all it's variants.
The quantum kicked rotor is defined by the Hamiltonian:

\begin{equation}
H = \frac{p^2}{2} + k cos(\theta) \sum_{n \in \mathbb{Z}} \delta(t - n\tau)
\end{equation}

This classical analog of this system reduces to the Chirikov standard map
\cite{stockmann}:

\begin{align}
    p_{n+1} = p_n + k sin(\theta_n) \\
    \theta_{n+1} = \theta_n + p_{n+1} \label{eq:classicalmap}
\end{align}

This map is chaotic and shows a great deal of complexity, but we will only
deal with the quantum version here. The key point for us, is that this
classical map displays diffusion in the angular momentum space. This can be
seen qualitatively by looking at \ref{eq:classicalmap}. When $p_n$ becomes
larger than $2\pi$, which will happen after a few kicks if $k$ is large,
then the successive $\theta_n$ will become uncorrelated. As the sign of
$sin(\theta_n)$ becomes random, the sequence ${p_n}$ describes a random walk.
This can be backed by a quantitative calculation which shows that the
the distribution $f_N(p)$ i.e. the distribution of momentum after N steps
wrt different initial conditions ($\theta_0$) is well approximated by a
gaussian distribution with a diffusion constant \cite{stockmann}:
\begin{equation}
    D = \frac{k^2}{2}
\end{equation}

As a time dependent problem, the quantum kicked rotor has no energy eigenstates.
However, since it is a periodic kicking potential, we can use floquet operators
to simulate it.

\begin{align}
    F &= \lim_{\epsilon \to 0} \int_{0}^{\tau + \epsilon} exp(-iHt/\hbar) dt\\
    &= exp\left(\frac{-i}{\hbar} cos\ \theta\right)
    exp\left(\frac{-i\tau}{2\hbar}p^2\right)
\end{align}

The floquet operator allows us to obtain the state of the system at $t = n\tau$.
Working in the eigenbasis of the (angular) momentum operator
$p \ket{n} = \hbar n \ket{n}; \braket{\theta | n} = e^{in\theta}$, we get:

\begin{equation}
    \braket{m | F | n} = exp\left(-\frac{i \tau}{2\hbar}m^2\right) i^{n-m}
    J_{n-m}\left(\frac{k}{\hbar}\right)
\end{equation}

We can use this expression for the floquet matrix to simulate the quantum
kicked rotor system as $F^N \ket{\psi(0)} = \ket{\psi(N\tau)}$. Even though,
the system doesn't have stationary states, we can obtain floquet eigenstates
by diagonalising the floquet operator.

An important variant of the kicked rotor system is the quasiperiodic kicked
rotor, which is discussed further in the report. It is given by the
hamiltonian:

\begin{equation}
H = \frac{p^2}{2} + \mathcal{K}(t) cos(\theta)\sum_{n \in \mathbb{Z}}
\delta(t - n\tau)
\end{equation}

where $\mathcal{K}(t) = k(1 + cos(\omega_2 t + \phi_2) cos(\omega_3 t
+ \phi_3))$.

\chapter{Readings and Work Done}
\section{Localisation}
The quantum kicked rotor also shows the same diffusion mechanism as the
classical analog in the beginning. If we take the initial state as $\ket{0}$
which has a uniform distribution of $\theta's$, then the system shows a
gaussian shape in the momentum space. But after a certain time, the
diffusion is suppressed by quantum effects which lead to an exponential
localisation in the momentum space distribution around $\ket{0}$. This
phenomenon is called `dynamical localisation'. It is analogous to the
Anderson localisation found in tight-binding systems.

\section{Anderson Localisation}
We now take a short detour to explain the phenomenon of Anderson Localisation.
The contents of this section are largely taken from \cite{muller_disorder_2016}.

Consider a non-relativistic particle or gaussian wave packet propogating
through a channel. If the channel has a constant potential throughtout
then the wave will propogate through it unimpeded performing ballistic
motion. If the channel has a constant potential with a small noise
term, then the potential landscape will look like a series of speckles
on an otherwise flat surface. Let us assume these speckles are well
separated and look more like spikes rather than shallow hills.

We have three length scales here, $l$ the
spacing between these speckles, $\delta$ the width of the speckle and
$\lambda$ the de Broglie wavelength of the particle. We assume $\delta <<
\lambda << l$ and thus, the particle sees the speckles as well-separated
$\delta$-spikes.

Each of these speckles - wells and peaks - acts as a
scatter for our particle. Quantum mechanically, each of them has a
finite, non-zero probability of both reflecting and transmitting the
particle.

Suppose then that our particle hits scatterer 1 and gets
transmitted with some probability. It then performs ballistic motion and
hits scatterer 2. Again there is some probability of transmission and
reflection. The particle may get transmitted immediately, or it may
undergo reflection twice (a complete internal reflection) and then get
transmitted. It may undergo a complete internal reflection multiple times
before crossing the scatterer. This leads to the net transmission
probability being dependent on the phase difference accumulated over
a complete internal reflection.

The net transmission probability can
be found by multiplying the transfer matrices of the two scatterers and
obtaining an overall transfer matrix. This yields the expression:

\begin{equation}
    T_{12} = \frac{T_1 T_2}{1 + \sqrt{R_1 R_2} e^{i\theta}}
\end{equation}
where $T_i$ and $R_i$ denote the transmission and reflection
probabilities of the $i$th scatterer, $\theta$ is the phase accumulated
in a complete internal reflection.

This phase is distributed randomly irrespective of the distribution of
distances between the scatterers as $\lambda << l$ and thus, we may
assume a uniform distribution for $\theta$.

Decoherence can occur if during the ballistic part, the particle couples
to an external degree of freedom. In such a scenario, we can assume a
uniform distribution for $\theta$ and obtain an average transmission
probability. In doing this, we are effectively killing any interference
effects as we are considering the phase to be completely scrambled by the time
it reaches scatterer 2. The calculations then yield $\langle T_{12} \rangle
= T_1 T_2 / (1 - R_1 R_2)$. For a channel of length $L$ and nearly uniform
scatterer density $n$, this gives us $R/T \propto nL$ which is Ohm's law.

But things
are much more interesting if we have phase-coherent transmission through
the channel. We must then find a quantity that is additive for such a
transmission and average over it. $\kappa = - ln\ T$ is such a quantity.
It exhibits the property $\langle ln\ T_{12} \rangle = ln\ T_1 + ln\ T_2$ and
due to this, it displays self-averaging.

Therefore, for a channel of
length $L$, $|\langle ln\ T \rangle| \propto nL$. Since $T < 1$, $ln\ T < 0$
and thus we have $exp(\langle ln\ T \rangle) \sim exp(-L/\xi_{loc})$ where
$\xi_{loc}$ is called the localisation length. This exponential localisation
in the absence of absorption is a hallmark of strong localisation.

Here,
$ln\ T$ exhibits a normal distribution and its peak corresponds to the most
likely value $T_{typ} = exp(\langle ln\ T \rangle) = e^{-L/\xi_{loc}}$. This
phenomenon is called Anderson localisation.

In 3D, the situation becomes
more complicated and the system can either allow transmission or show
localisation based on disorder strength. These two regimes are separated by
a 2nd order phase transition known as the Anderson transition or the
metal-insulator transition, metal referring to transmission and insulator
referring to localisation of the particle.

Even though Anderson originally
introduced it in the form of a tight-binding model with discrete spatial
sites, our presented model also exhibits the same behaviour and thus has
been used in this report.

\section{Robustness}

\section{Anderson Transition}

\section{Bipartite Entanglement}

\section{Spectral Analysis}

\chapter{Epilogue}
\section{Future Prospects}

\section{Conclusion}

\printbibliography

\end{document}
