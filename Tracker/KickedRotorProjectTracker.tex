\documentclass[12pt]{article}


\usepackage{braket}
\usepackage{amsmath}
\usepackage{ulem}
\usepackage{biblatex}
\addbibresource{Readings.bib}

%opening
\title{Quantum Kicked Rotor}
\author{Aditya Chincholi}

\begin{document}

\maketitle

\section{Introduction}

\section{Questions}
\begin{enumerate}
    \item What does kicking the rotor periodically have anything to do with a random walk?
    
    \item What is the analogy with Anderson localization? After all, Anderson localisation
    is about a diffusing wavefunction which encounters disorder in the form of passive scatterers
    and gets reflected/transmitted with certain probability. This transmission amplitude goes down
    exponentially with length of the sample. In contrast, in the kicked rotor we have an initial
    condition of uniform distribution in the position space. We have active ``kicks" which pump
    energy into the system and these kicks strengths are pseudo-random apparently.
    
    \item \label{ques:trulyrandom} If we take a quantum rotor and kick it with a truly random kick strength like $Ka_j$ at $t = j$
    where $a_j = 1$ with probability $p$ and $a_j = -1$ with probability $1-p$, would it also show initial
    diffusion and subsequent localisation?
    
    \item In the Anderson localisation (at least for d=1 case), we see the ``transport" being matter
    transport i.e. we comment on the chances that the matter particle is transmitted across the sample.
    One could also say that the energy of the particle (which is constant) gets transported to position
    states far from the origin. But in the case of the kicked rotor, what exactly is being transported?
    Energy is being pumped into and taken out of this system at all levels, so what is being transported?
    
    \item \label{ques:timestepsbound} In the simulation, given a particular dimension of the fourier space
    of $\ket{\psi}$, how do we get a bound on the maximum timestep?
    
\end{enumerate}

\section{Partial Answers}
\begin{enumerate}
    \item Let us try and answer the question \ref{ques:trulyrandom}. First consider a hamiltonian given by:

    \begin{equation}
        H = \frac{L^2}{2} + \hbar K \sum_{j=0}^{\infty} a_j \delta(t - j)
    \end{equation}

    where $P(a_j = 1) = p, P(a_j = -1) = 1 - p$. We can take $p = \frac{1}{2}$ for simplicity. We then have
    the unitary operator $U_j$ to evolve the state from after the (j-1)th kick to after the jth kick.
    
    \begin{align}
        U_j &= exp(-iKa_j) exp(-i\frac{L^2}{2\hbar}) \\
        U_j \ket{m} &= exp(-i (Ka_j + \frac{\hbar m^2}{2})) \ket{m}	
    \end{align}
    where $\ket{m}$ is the eigenstate of angular momentum operator $L$. We can see clearly here that this
    random ``kick" actually does nothing. It doesn't project our system from one angular momentum eigenstate
    to another. So this is just a phase shift of each existing eigenstate. No new $\ket{m}$ states can be
    occupied which weren't occupied before. Clearly, the disorder $\leftrightarrow$ random kick strength
    analogy fails in this respect. Lets try the following general hamiltonian and find the problem.

    \begin{equation}
        H = \frac{L^2}{2} + \hbar K V(\theta) \sum_{j=0}^{\infty} a_j \delta(t - j)
    \end{equation}
    
    Then we get
    \begin{align}
        U_j &= exp(-iKa_jV(\theta)) exp(-i\frac{L^2}{2\hbar}) \\
        U_j \ket{m} &= 	\frac{e^{-i \hbar m^2/2}}{\sqrt{2\pi}} \int exp(-iKa_jV(\theta)) exp(-im\theta) \ket{\theta} d\theta \\
        &= 	\frac{e^{-i \hbar m^2/2}}{\sqrt{2\pi}} \int exp(-i(Ka_j\frac{V(\theta)}{\theta} + m)\theta) \ket{\theta} d\theta \\
        &[\text{And if we take } V(\theta) = \theta] = e^{-i \hbar m^2/2} \ket{m + Ka_j}
    \end{align}
    
    So clearly, $V(\theta)$ needs at least a $\theta$ term in order to kick the system into other states. The
    problem is that without theta dependence, the initial m term will never break into pieces which hop to other
    states. \sout{The issue is one cannot think of the state hopping from $\ket{m}$ to $\ket{n}$, rather one must look at it from the $\theta$ space perspective.}
    
    \item A naive answer to question \ref{ques:timestepsbound} might be to do the following energy calculation:
    \begin{align}
        T k &= \frac{1}{2} \hbar^2 L_{max}^2 \\
        T &\leq \frac{\hbar^2 L_{max}^2}{2 k} \\
        [\text{If we take } \hbar &= 1, L_{max} = 1000, k = 5] \nonumber \\
        T &\leq 10^5
    \end{align}
    which undoubtedly seems like an overestimate.
\end{enumerate}

\section{Week 7}
Deadline: Next Saturday (10 April 2021)
\subsection{Objectives}
\begin{enumerate}
    \item Work out the $a_j$ model analytically and computationally both.
    
    \item Add noise to kick period in the kicked rotor:
    Kick the rotor at $\tau \pm \delta\tau$ where $\delta\tau$ is drawn from a uniform distribution.
    Loss of localization is expected.
    
    \item Add noise to kick strength in the kicked rotor:
    Kick the rotor with strength $k \pm \delta k$ where $\delta k$ is from uniform distribution.
    Loss of localization is expected.
    
    \item Read up on the quasi-periodic kicked rotor and the metal-insulator transition in it.
\end{enumerate}

\subsection{Randomly Kicked Rotor}
\begin{equation}
    
\end{equation}


\nocite{*}
\printbibliography

\end{document}
